%
% 総合研究論文サンプルファイル
%
% ・和文題名, 英文題名, 学籍番号, 姓, 名, 指導教員名, 職位 を記入する
%   (====== で囲まれた範囲).
% ・顔写真 (履歴書の余りなど) を用意し, 所定の位置に貼る.
%   デジタルデータ (JPG 等) があれば, それを使ってもよい (下記設定変更).
%
\documentclass[a4j,12pt]{jarticle}
\usepackage{amsmath}
\usepackage{amssymb}
\usepackage[dvipdfmx]{graphicx}%画像挿入
\usepackage{theorem}%定義や定理の環境
\usepackage{url}
%必要なパッケージを追加



%========================================
% 各自の内容を記述
%
\def\YEAR{20XX}  % 総合研究着手年度
\def\TITLE{フェルマーの定理と私}  % 和文題名
\def\ETITLE{Fermat's theorem and me}  % 英文題名
\def\NUMBER{BV?????}  % 学籍番号
\def\FAMILYNAME{\ruby{数理}{すうり}}  % 姓
\def\FIRSTNAME{\ruby{科太郎}{かたろう}}  % 名
\def\ADVISER{大宮 シス子}  % 指導教員 氏名
\def\POSITION{教授}  % 指導教員 職位
\def\JPG{SIT.png}  % 顔写真ファイル名 (なければ空白で可)
%
% 顔写真の設定: 下のいずれかの行を活かし, 他方をコメントアウト
%
\def\FACE{\WAKU}   % 枠線のみ表示
%\def\FACE{\IMAGE}  % 顔写真ファイル使用時
%========================================
% ルビの定義
%
\def\ruby#1#2{%
\leavevmode
\setbox0=\hbox{#1}\setbox1=\hbox{\scriptsize#2}%
\ifdim\wd0>\wd1 \dimen0=\wd0 \else \dimen0=\wd1 \fi
\hbox{\kanjiskip=\fill
 \vbox{\hbox to \dimen0{\scriptsize \hfil#2\hfil}%
 \nointerlineskip
 \hbox to \dimen0{\hfil#1\hfil}}}}
%
% 写真の定義
\def\WAKU{\framebox[30mm]{\rule{0mm}{38mm}\raisebox{18mm}{\smash{
 \parbox{25mm}{\begin{center}顔写真\\ 横3cm\\ ×\\ 縦4cm\end{center}}}}}
 \\[7mm]}
\def\IMAGE{\parbox{33mm}{\rule{0mm}{38mm}\raisebox{18mm}{\smash{
 \parbox{31mm}{\includegraphics[height=40mm,keepaspectratio]{\JPG}}}}}
 \\[12mm]}
%----------------------------------------
\begin{document}
\thispagestyle{empty}
\begin{center}
 \YEAR{}年度 \\
 芝浦工業大学\quad システム理工学部 \\
 数理科学科 \\[12mm]
 \huge 総合研究論文
\end{center}
\vspace{16mm}\par
%
% 題名
%
\noindent\smash{
 \begin{minipage}[t]{.98\linewidth}
  \begin{center}
   \Huge\bf\TITLE  % 和文題名
   \\[1ex]
   \Large\bf\ETITLE  % 英文題名
  \end{center}
 \end{minipage}}
\vspace{47mm}\par
%
% 顔写真, 学籍番号, 氏名, 指導教員
%
\begin{center}
 \FACE                                        % 顔写真
 \Large \NUMBER                     \\[7mm]   % 学籍番号
 \huge  \FAMILYNAME\quad \FIRSTNAME \\[15mm]  % 氏名
 \Large 指導教員: \ADVISER ~ \POSITION
\end{center}
\vfill
\newpage
\setcounter{page}{1}
\tableofcontents	%目次
\thispagestyle{empty}
\newpage
%%%%%%↓ここから本文↓%%%%%%
\section{はじめに}
\newpage
%
\section{つぎに}
\subsection{小節のタイトル}
\subsubsection{小小節のタイトル}
\newpage
%
\section{おわりに}

\newpage

%参考文献
\begin{thebibliography}{99}
\bibitem{a}【本の場合】著者名,タイトル,出版社,出版年.
\bibitem{b}【論文の場合】著者名,タイトル,雑誌名,巻・号,出版年度,頁.
\bibitem{c}【Webページの場合】 タイトル,ページ制作者(機関)等,URL: \url{http://www.shibaura-it.ac.jp/},最終アクセス日時: 2021/12/28 16:33.
\end{thebibliography}

\end{document}
